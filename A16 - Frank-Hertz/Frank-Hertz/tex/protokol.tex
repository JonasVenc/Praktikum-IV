% ----------------------------------------------------------------------
%  Pracovní úkoly
% ----------------------------------------------------------------------
\section{Pracovní úkoly}

\begin{enumerate}
\item Změřte charakteristiky Franck-Hertzovy trubice s parami rtuti při teplotách baňky $t_1$=70 °C, $t_2$=110 °C, $t_3$=170 °C. Pro nastavení vhodných parametrů $U_H$, $U_1$, $U_2$ sledujte charakteristiky na osciloskopu.

\item Z naměřených závislostí určete rezonanční potenciál atomů rtuti a vlnovou délku odpovídající rezonančnímu přechodu.

\item Změřte charakteristiku Franck-Hertzovy trubice s neonem až do 99 V. Při měření sledujte, jak se při nárůstu napětí $U_1$ mění počet svítících vrstev v prostoru mezi mřížkami. Pomocí manuálního nastavení napětí $U_1$ určete závislost počtu svítících vrstev na jeho velikosti.

\item Určete rezonanční potenciál atomů neonu a vlnovou délku odpovídající rezonančnímu přechodu.

\end{enumerate}

% ----------------------------------------------------------------------
%  Teoretická část
% ----------------------------------------------------------------------
\section{Teoretická část}

% ----------------------------------------------------------------------
%  Výsledky a zpracování měření
% ----------------------------------------------------------------------
\section{Výsledky a zpracování měření}

\subsection{Páry rtuti}

Nejprve jsme proměřili volt-ampérovou charakteristiku par rtuti při teplotách 70 °C, 110 °C a 170 °C. Tyto závislosti jsou znázorněny na obrázcích.




    
% ----------------------------------------------------------------------
%  Diskuse výsledků
% ----------------------------------------------------------------------			
\section{Diskuse výsledků}

% ----------------------------------------------------------------------
%  Závěr
% ----------------------------------------------------------------------
\section{Závěr}
